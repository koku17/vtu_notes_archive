\documentclass{article}

% pacakges
\usepackage[margin=.5in]{geometry}
\usepackage{enumitem}
\usepackage{xcolor}
\usepackage{hyperref}
\usepackage{chemfig}
\usepackage{amsmath}
\usepackage{array}
\usepackage{multirow}

% detailes
\author{koku17}
\title{INTRODUCTION TO BIOLOGY}

% presets
\pagecolor{black}
\color{white}
\hypersetup{colorlinks=true,linkcolor=white}

% macros
\newcommand{\tbox}[1]{
	\begin{tabular}{l}
		#1
	\end{tabular}
}

\begin{document}
	\maketitle \thispagestyle{empty} \newpage
	\tableofcontents \thispagestyle{empty} \newpage \setcounter{page}{1}
	\section{The cell}
	\subsection{the basic unit of life}
	\subsection{Structure and functions of a cell}
	\subsection{The Plant Cell and animal cell}
	\subsection{Prokaryotic and Eukaryotic cell}
	\subsection{Stem cells and their application}

	\section{Biomolecules}
	\textbf{Carbohydrates} \\
	Carbohydrates are made up of Carbon (C), Oxygen(O) and Hydrogen(H) in the ration $1:2:1$ i.e
		$\text{C}_n\text{H}_{2n}\text{O}_n$

	\subsection{Properties and functions of Carbohydrates}
	\subsubsection{Properties of Carbohydrates}
	\begin{center}
		\begin{tabular}{|c|p{.6133\columnwidth}|} \hline
			Properties & \multicolumn{1}{c|}{Description} \\ \hline
			Chemical Composition & Carbon, Hydrogen and Oxygen in the ratio 1:2:1 \\ \hline
			Structure & Carbon Chain or Ring \\ \hline
			Type & Monosaccharides, Disaccharides and Polysaccharides \\ \hline
			\multirow{2}{*}{Solubility} & Most of the carbohydrates are soluble in water, except
				large carbohydrates \newline Example : Glucose, Fructose \\ \hline
			\multirow{2}{*}{Hydrogen Bonding} & Carbohydrate form hydrogen bonds with other
				molecules \newline Without it carbohydrate cannot soluble in water \\ \hline
			\multirow{2}{*}{Sweetness} & They are sweet for Monosaccharides, Disaccharides \newline
				Tasteless for Polysaccharides \\ \hline
			\multirow{2}{*}{Energy Source} & Carbohydrates are the primary source of energy in
				living organisms \newline Burning 1g of carbohydrate produces 4 cal \\ \hline
			Structural Function & Acts as Structural support in plants as cellulose and
				Exoskeleton as in animals \\ \hline
			Biological Significance & Used in cell signaling, immune system functions \\ \hline
		\end{tabular}
	\end{center}

	\subsubsection{Functions of the Carbohydrate}
	\begin{itemize}
		\item \textbf{Structural Support}
			\begin{itemize}
				\item Cellulose is a complex carbohydrate offers rigidity and strength,
					facilitate  digestion
				\item Fiber is a type of carbohydrates, found in plant based carbohydrate
					supports healthy digestive systems
				\item It acts as medium of cell communication and cell recognition, which
					enables immune response and tissue development
				\item It acts as a fuel for exercises, when there is a necessity of energy
					muscle contraction \& endurance takes place provided by carbohydrates
			\end{itemize}
	\end{itemize}

	\subsection{Nucleic acids}
	\begin{itemize}
		\item They include DNA, RNA
			\begin{itemize}
				\item \textbf{DNA} \\
					It carries genetic information, encodes the information required for
					growth, development and reproduction of organism
				\item \textbf{RNA} \\
					It is used for protein synthesis, gene expression and regulation
			\end{itemize}
		\item Both DNA and RNA contribute energy transfer [in ATP (Adenosine Triphosphate)], catalyzes
			biolytical reactions and respond to immune system
	\end{itemize}

	\subsubsection{Properties of Nucleic acids}
	\begin{center}
		\begin{tabular}{|c|p{.7\columnwidth}|} \hline
			Polarity & Nucleic acid contains 5 \& 3 end phosphate and hydroxyl group which causes
				polarity \\ \hline
			\multirow{8.5}{*}{Structure} & \textbf{DNA} \vspace{.5em} \newline
				\begin{minipage}{.7\columnwidth}
					\begin{itemize}[itemsep=0pt,itemindent=-2.5em,label=]
						\item Double helix to protect the genetic information encoded
						\item It helps for the replication and transcription
						\item Hydrogen bond occurs between complementary bases which
							provides structural integrity
						\item \vspace{-.5em} \& specificity to the molecule
					\end{itemize}
				\end{minipage} \vspace{.5em} \newline
				\textbf{RNA} \newline
				Single stranded structured hydrogen bonding \\ \hline
			Acidic nature & Due to the presence of phosphate group \\ \hline
		\end{tabular}
	\end{center}

	\subsubsection{Functions of Nucleic acids}
	\begin{itemize}
		\item Genetic information storage
			\begin{itemize}
				\item In living organism DNA is a primary source of genetic information storage
				\item Help in growth of the cells \& its functions and reproduction
			\end{itemize}
		\item Protein synthesis
			\begin{itemize}
				\item RNA is involved in this process, which includes mRNA, tRNA and rRNA
					\newline Which transfers genetic information from DNA by transcription
					and translation
				\item Nucleic acid especially ATP functions as the carrier of energy as well as
					storage of energy
				\item During cellular metabolism ATP is broken down, provides necessary energy
					for cellular activities
			\end{itemize}
	\end{itemize}

	\subsection{proteins}
	\subsection{lipids}
	\subsection{Importance of special biomolecules}

	\section{Enzyme}
	\subsection{Classification with (one example each)}
	\subsection{Properties and functions}
	\subsection{vitamins and hormones}
\end{document}
