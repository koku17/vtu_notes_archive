\documentclass{article}

% packages
\usepackage[margin=.5in]{geometry}
\usepackage{enumitem}
\usepackage{amsmath,amssymb}
\usepackage{xcolor}
\usepackage{hyperref}

% detailes
\author{koku17}
\title{Mathematical Logic}

% presets
\pagecolor{black}
\color{white}
\hypersetup{
	colorlinks=true,
	linkcolor=white
}

\begin{document}
	\maketitle \thispagestyle{empty} \newpage
	\tableofcontents \thispagestyle{empty} \newpage
	\setcounter{page}{1}
	\section{Basic Connectives and Truth Tables}
	\subsection{Definition}
	\subsubsection{Logic}
	It is the science dealing with the methods of reasoning

	\subsubsection{Propositions}
	\begin{itemize}
		\item It is a statement, which in a given context, can be to either true or false, but not both
		\item Proposition are usually denoted by small letters such as $p,q,r,s,\cdots$
		\item [] \textbf{Example} \\
			$p : \text{Banglore is in Karnataka}$ \\
			$q : \text{2 is a prime number}$
	\end{itemize}

	\subsubsection{Truth value}
	\begin{itemize}
		\item The truth or falsity of a proposition is called it's truth value
		\item If a proposition is true, we will indicate the truth value by \textbf{1} and
			if it is false denoted by the value \textbf{0}
		\item [] \textbf{Example} \\
			$p : \text{3 is a prime number}$ (The value for $p$ is 1) \\
			$q : \text{Every rectangle is a square}$ (The value for $q$ is 0)
	\end{itemize}

	\subsection{Logical connectives \& Truth tables}
	\begin{itemize}
		\item Words or phrases like \textbf{not}, \textbf{and}, \textbf{or}, \textbf{if then}
			and \textbf{if and only if} are called Logical Connectives
		\item The new propositions obtained by the use of connectives are called Compound Propositions
		\item The original propositions, from which a compound proposition is obtained are called
			Components or Primitives of the compound propositions
		\item Proposition, which do not contain any logical connectives are called Simple Propositions
	\end{itemize}

	\subsubsection{Negation $(\neg)$}
	\begin{itemize}
		\item A proposition obtained by inserting the word \textbf{not} at an appropriate place in a
			given proposition is called negation of the given proposition
		\item It is denoted by $\neg$, read as not $p$
		\item If the truth value of proposition $p$ is 1, then truth table value of it's negation is 0
			and if the truth value of $p$ is 0, then the truth value of it's negation is 1
			\begin{center}
				Truth Table \\ \vspace{1em}
				\begin{tabular}{|c|c|} \hline
					$p$ & $\neg p$ \\ \hline
					1 & 0 \\
					0 & 1 \\ \hline
				\end{tabular}
			\end{center}
		\item [] \textbf{Example} \\
			$p : \text{3 is a prime number}$ \\
			$q : \text{3 is \textbf{not} a prime number}$
	\end{itemize} \newpage

	\subsubsection{Conjunction $(\land)$}
	\begin{itemize}
		\item A compound proposition obtained by combining two given proposition by inserting the word
			\textbf{and} in between them, is called Conjunction
		\item It is denoted by $\land$, i.e $p\land q$ read as $p$ and $q$
		\item The conjunction $p\land q$ is true only when both $p$ \& $q$ are true, otherwise it is
			false
			\begin{center}
				Truth Table \\ \vspace{1em}
				\begin{tabular}{|c|c|c|} \hline
					$p$ &  $q$ & $p\land q$ \\ \hline
					0 & 0 & 0 \\
					0 & 1 & 0 \\
					1 & 0 & 0 \\
					1 & 1 & 1 \\ \hline
				\end{tabular}
			\end{center}
		\item [] \textbf{Example} \\
			$p : \sqrt{2}\text{ is irrational number (1)}$ \\
			$q : 9\text{ is prime number (0)}$ \\
			$p\land q=0$
	\end{itemize}

	\subsubsection{Disjunction $(\lor)$}
	\begin{itemize}
		\item A compound proposition obtained by combining two given proposition by inserting the word
			\textbf{or} in between them, is called Disjunction
		\item It is denoted by $\lor$, i.e $p\lor q$ read as $p$ and $q$
		\item The disjunction $p\lor q$ is false only when both $p$ \& $q$ are false, otherwise it is
			true
			\begin{center}
				Truth Table \\ \vspace{1em}
				\begin{tabular}{|c|c|c|} \hline
					$p$ &  $q$ & $p\land q$ \\ \hline
					0 & 0 & 0 \\
					0 & 1 & 1 \\
					1 & 0 & 1 \\
					1 & 1 & 1 \\ \hline
				\end{tabular}
			\end{center}
		\item [] \textbf{Example} \\
			$p : \sqrt{2}\text{ is irrational number (1)}$ \\
			$q : 9\text{ is prime number (0)}$ \\
			$p\lor q=1$
	\end{itemize}

	\subsubsection{Exclusive Disjunction $(\veebar)$}
	\begin{itemize}
		\item A compound proposition obtained by combining two given proposition by inserting the word
			\textbf{or} in between them, is called Exclusive Disjunction
		\item The disjunction $p\veebar q$ is true only either $p$ or $q$ is true, but not both
			\begin{center}
				Truth Table \\ \vspace{1em}
				\begin{tabular}{|c|c|c|} \hline
					$p$ &  $q$ & $p\veebar q$ \\ \hline
					0 & 0 & 0 \\
					0 & 1 & 1 \\
					1 & 0 & 1 \\
					1 & 1 & 0 \\ \hline
				\end{tabular}
			\end{center}
		\item [] \textbf{Example} \\
			$p : \sqrt{2}\text{ is irrational number (1)}$ \\
			$q : 3\text{ is prime number (1)}$ \\
			$p\veebar q=0$
	\end{itemize}

	\subsubsection{Conditional $(\to)$}
	\begin{itemize}
		\item A compound proposition obtained by combining two given proposition by using the words
			\textbf{if then} is called Conditional
		\item For a given two proposition $p$ \& $q$
			\begin{itemize}
				\item if $p$ then $q$ is denoted by $p\to q$
				\item The conditional $p\to q$ is false, only when $p$ is true and $q$ is
					false, otherwise true
			\end{itemize}
			\begin{center}
				\begin{tabular}{|c|c|c|} \hline
					$p$ & $q$ & $p\to q$ \\ \hline
					0 & 0 & 1 \\
					0 & 1 & 1 \\
					1 & 0 & 0 \\
					1 & 1 & 1 \\ \hline
				\end{tabular}
			\end{center}
		\item [] \textbf{Example} \\
			$p : \text{2 is prime number (1)}$ \\
			$p : \text{4 is prime number (0)}$ \\
			$p\to q=0$
	\end{itemize}

	\subsubsection{BiConditional $(\leftrightarrow)$}
	\begin{itemize}
		\item A compound proposition obtained by combining two given proposition by using the words
			\textbf{if and only if} is called BiConditional
		\item For a given two proposition $p$ \& $q$,
			\begin{itemize}
				\item if $p$ then $q$ and if $q$ then $p$, is denoted by $p\leftrightarrow q$
					or $(p\to q)\land(q\to p)$
			\end{itemize}
		\item The bi-conditional $p$ if and only if $q$ is true, if it is only when both $p$ and $q$
			are true or both $p$ and $q$ are false
			\begin{center}
				\begin{tabular}{|c|c|c|c|c|} \hline
					$p$ & $q$ & $p\to q$ & $q\to p$ & $p\leftrightarrow q$ \\ \hline
					0 & 0 & 1 & 1 & 1 \\
					0 & 0 & 1 & 0 & 0 \\
					1 & 0 & 0 & 1 & 0 \\
					1 & 1 & 1 & 1 & 1 \\ \hline
				\end{tabular}
			\end{center}
		\item [] \textbf{Example} \\
			$p : \text{2 is prime number (1)}$ \\
			$p : \text{4 is prime number (0)}$ \\
			$p\leftrightarrow q=0$
	\end{itemize}

	\section{Logic Equivalence $-$ The Laws of Logic}

	\section{Logical Implication $–$ Rules of Inference}

	\section{The Use of Quantifiers}
	\section{Quantifiers}
	\section{Definitions and the Proofs of Theorems}
\end{document}
