\documentclass{article}

% packages
\usepackage[margin=.5in]{geometry}
\usepackage{enumitem}
\usepackage{tikz}
\usepackage{algorithmic}
\usepackage{amsmath,amssymb}

% detailes
\author{koku17}
\title{Module 1}

% presets

% macros

\begin{document}
	\maketitle \thispagestyle{empty} \newpage
	\tableofcontents \thispagestyle{empty} \newpage \setcounter{page}{1}
	\section{INTRODUCTION}
	

	\subsubsection*{Application of Algorithms}
	\begin{itemize}
		\item \textbf{Computer Programming} \\
			Algorithm is used in Computer programming to solve complex problems efficiently like, Sorting,
			Searching and manuplating data structures.
		\item \textbf{Artificial Intelligence} \\
			Algorithm plays critical role in Artificial Intelligence for tasks like Computer vision,
			Speech Recognition and Natural Language understanding.
		\item \textbf{Cryptography} \\
			Algorithm is used in Cryptography such as RSA and AES, which are used to secure data transmission
			and storage.
		\item \textbf{Scientific Computing} \\
			Algorithms are Scientific computing to develop algorithms for tasks such as numerical integration,
			optimization and simulation.
		\item \textbf{Game Development} \\
			Algorithms are used in Game development for tasks such as pathfinding, collision detection, and
			physics simulation.
		\item \textbf{Big data processing} \\
			Algorithm is used in Data processing for tasks such as Data Mining, Machine learning, and
			Natural Language Processing.
	\end{itemize}

	\subsection{What is an Algorithm ?}
	An algorithm is a sequence of unambiguous instructions for solving a problem, i.e., for obtaining a required
	output for any legitimate input in a finite amount of time.

	\begin{itemize}
		\item The nonambiguity requirement for each step of an algorithm cannot be compromised.
		\item The range of inputs for which an algorithm works has to be specified carefully.
		\item The same algorithm can be represented in several different ways.
		\item There may exist several algorithms for solving the same problem.
		\item Algorithms for the same problem can be based on very different ideas and can solve the problem
			with dramatically different speeds.
	\end{itemize}

	\subsubsection{Euclid's algorithm}
	\begin{enumerate}[label=\textbf{Step \arabic* :},itemindent=4em]
		\item If $n = 0$, return the value of m as the answer and stop otherwise, proceed to Step 2.
		\item Divide $m$ by $n$ and assign the value of the remainder to $r$.
		\item Assign the value of $n$ to $m$ and the value of $r$ to $n$, Go to Step 1.
	\end{enumerate}

	\subsubsection{Algorithm}
	\begin{algorithmic}
		\PROCEDURE {}
		\WHILE {$n\ne0$ do}
			\STATE $r\gets m\mod n$
			\STATE $m\gets n$
			\STATE $n\gets r$
		\ENDWHILE
		\RETURN $m$
	\end{algorithmic}

	\subsection{Fundamentals of Algorithmic Problem Solving}

	\section{FUNDAMENTALS OF THE ANALYSIS OF ALGORITHM EFFICIENCY}
	\subsection{Analysis Framework}
	\subsection{Asymptotic Notations and Basic Efficiency Classes}
	\subsection{Mathematical Analysis of Non recursive Algorithms}
	\subsection{Mathematical Analysis of Recursive Algorithms}

	\section{BRUTE FORCE APPROACHES}
	\subsection{Selection Sort and Bubble Sort}
	\subsection{Sequential Search}
	\subsection{Brute Force String Matching}
\end{document}
