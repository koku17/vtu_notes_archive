\documentclass{article}

% packages
\usepackage[margin=.5in]{geometry}
\usepackage{enumitem}
\usepackage{hyperref}
\usepackage{xcolor}
\usepackage{pgffor}
\usepackage{calc}

% detailes
\title{Module 1}
\author{koku17}

% presets
\pagecolor{black}
\color{white}
\hypersetup{
	colorlinks=true,
	linkcolor=white
}

% macro
\newcounter{linecount}
\newcommand{\Newline}[1]{
	\setcounter{linecount}{#1-1}
	\foreach \i in {1,...,\value{linecount}}{\\ \null \\}
}

\begin{document}
	\maketitle \thispagestyle{empty} \newpage
	\tableofcontents \thispagestyle{empty} \newpage
	\section{Introduction to Databases}
	\subsection{Introduction}
	\subsubsection{Definition}
	\textbf{Database} \\
	Collection of related data with an implicit meaning is a database \\
	The DBMS is a general-purpose software system that facilitates the processes of defining, constructing,
	manipulating, and sharing databases among various users \Newline{2}
	\textbf{Data} \\
	A data mean known facts that can be recorded and that have implicit meaning \Newline{2}
	\textbf{Example}
	\begin{itemize}[label=]
		\item name list
		\item telephone numbers
		\item addresses of the people
	\end{itemize}

	\subsubsection{Implicit Properties of DBMS}
	\begin{itemize}
		\item A database represents some aspect of the real world, sometimes called the miniworld or
			the universe of discourse (UoD)
		\item A database is a logically coherent collection of data with some inherent meaning \\
			A random assortment of data cannot correctly be referred to as a database
		\item A database is designed, built, and populated with data for a specific purpose \\
			It has an intended group of user sand some preconceived applications in which these
			users are interested
	\end{itemize}

	\subsubsection{Other Properties of DBMS}
	\begin{itemize}
		\item A database can be of any size and complexity
		\item A database may be generated and maintained manually or it may be computerized \\
			A computerized database may be created and maintained either by a group of application
			programs written specifically for that task or by a database management system
		\item
	\end{itemize}

	\subsection{Characteristics of database approach}
	\subsection{Advantages of using the DBMS approach}
	\subsection{History of database applications}

	\section{Overview of Database Languages and Architectures}
	\subsection{Data Models}
	\subsection{Schemas, and Instances}

	\section{Three schema architecture and data independence}
	\subsection{database languages, and interfaces}
	\subsection{The Database System environment}

	\section{Conceptual Data Modeling using Entities and Relationships}
	\subsection{Entity types}
	\subsection{Entity sets and structural constraints}
	\subsection{Weak entity types}
	\subsection{ER diagrams}
	\subsection{Specialization and Generalization}
\end{document}
