\documentclass{article}

% packages
\usepackage{xcolor}
\usepackage[margin=.5in]{geometry}
\usepackage{hyperref}
\usepackage{tikz}

% detailes
\author{koku17}
\title{Theory of Computation}

% presets
\hypersetup{
    hidelinks
}
\pagecolor{black}
\color{white}
\usetikzlibrary{
	automata,
	positioning,
	arrows.meta
}

% macros
\newcommand{\module}[1]{
	\phantomsection
	\begin{center}
		\textbf{{\LARGE#1}}
	\end{center}
	\stepcounter{section}
	\addcontentsline{toc}{section}{#1}
}

\begin{document}
    \pagenumbering{gobble} \maketitle \newpage
    \pagenumbering{roman} \pdfbookmark[1]{Contents}{} \tableofcontents \newpage
    \pagenumbering{arabic}

	\module{Module 1}
	\subsection{Introduction to Finite Automata}
	Automata theory is also known as Theory of Computation.
	Processing the input towards the output, every device has a low level calculating machine called as
	Abstract Machine model.
	These models we call as an Automata.
	If automata models are implemented in finite number of states it is called as Finite Automata.

	\subsubsection{Structural Representations}

	\subsubsection{Automata and Complexity}
	
	\subsection{The Central Concepts of Automata Theory}
	\subsubsection{Deterministic Finite Automata}
	\subsubsection{Nondeterministic Finite Automata}
	
	\subsection{An Application}
	\subsubsection{Text Search}
	\subsubsection{Finite Automata with Epsilon-Transitions}
	\newpage

	\module{Module 2}
	\subsection{Regular Expressions}
	\subsubsection{Finite Automata and Regular Expressions}
	\subsubsection{Proving Languages not to be Regular}
	\subsubsection{Closure Properties of Regular Languages}
	\subsubsection{Equivalence and Minimization of Automata}
	\subsubsection{Applications of Regular Expressions}
	\newpage

	\module{Module 3}
	\subsection{Context-Free Grammars}
	\subsubsection{Parse Trees}
	\subsubsection{Ambiguity in Grammars and Languages}
	\subsubsection{Definition of the Pushdown Automaton}
	\subsubsection{The Languages of a PDA}
	\subsubsection{Equivalence of PDA's and CFG's}
	\subsubsection{Deterministic Pushdown Automata}
	\newpage

	\module{Module 4}
	\subsection{Normal Forms for Context-Free Grammars}
	\subsubsection{The Pumping Lemma for Context-Free Languages}
	\subsubsection{Closure Properties of Context-Free Languages}
	\newpage

	\module{Module 5}
	\subsection{Introduction to Turing Machines}
	\subsubsection{Problems That Computers Cannot Solve}
	\subsubsection{The Turing Machine}
	\subsubsection{Programming Techniques for Turing Machines}
	\subsubsection{Extensions to the Basic Turing Machine}
	\subsubsection{Undecidability $-$ A Language That Is Not Recursively Enumerable}
\end{document}
