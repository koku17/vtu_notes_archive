\documentclass{article}

% packages
\usepackage[margin=.5in]{geometry}
\usepackage{enumitem}
\usepackage{hyperref}
\usepackage{newtx}
\usepackage{tikz}
\usepackage{xcolor}
\usepackage{float}
\usepackage{tocloft}
\usepackage[outputdir=tmp]{minted}

% details
\author{koku17}
\title{USP Model Paper}

% ricing
\def \darkmode{1}
\if1\darkmode
	\pagecolor{black}
	\color{white}
	\usemintedstyle{one-dark}
\else
	\usemintedstyle{gruvbox-light}
\fi

% presets
\hypersetup{
	hidelinks
}
\usetikzlibrary{
	arrows.meta,
	positioning,
	shapes.geometric
}

% macros
\def \OR{
	\item [] \phantomsection \begin{center} OR \end{center}
}

\def \question#1#2#3{
	\item [#1.] \phantomsection #2
	\ifx~#3
		\addcontentsline{toc}{subsection}{#1. #2}
	\else
		\addcontentsline{toc}{subsection}{#1. #3}
	\fi
}

\def \answer{\item[$\to$]}
\def \option#1{\mintinline{text}{#1}}
\def \output#1{{\color{gray}\inputminted{text}{#1}}}

\def \bhead#1{
	\begin{center}
		\phantomsection #1
		\addcontentsline{toc}{section}{#1}
	\end{center}
}

\begin{document}
	\pagenumbering{gobble} \maketitle \newpage
	\pagenumbering{roman} \tableofcontents \newpage
	\pagenumbering{arabic}
	\begin{enumerate}
		\question{1a}{With the example, explain the following commands}{~}
			\begin{enumerate}[label=\roman*)]
				\item \mintinline{c}{man}
				\item \mintinline{c}{pwd}
				\item \mintinline{c}{od}
				\item \mintinline{c}{cal}
				\item \mintinline{c}{date}
			\end{enumerate}
		\answer
			\begin{enumerate}[label=\roman*)]
				\item \mintinline{bash}{man ls} \output{output/man_ls}
					\begin{itemize}
						\item man  is  the  system's  manual  pager
						\item Each page argument given to man is normally the name of a program, utility or
							function
						\item The manual page associated with each of these arguments is then found and
							displayed
						\item A section, if provided, will direct man to look only in that section of the manual
						\item The default action is to search in all of the available sections following a pre
							defined order, and to show  only the first page found
								\begin{table}[H]
									\centering
									\begin{tabular}{cl}
										1 & Executable programs or shell commands \\
										2 & System calls (functions provided by the kernel) \\
										3 & Library calls (functions within program libraries) \\
										4 & Special files (usually found in \texttt{/dev}) \\
										5 & File formats and conventions, e.g. \texttt{/etc/passwd} \\
										6 & Games \\
										7 & Miscellaneous (including macro packages and conventions),
											e.g. \texttt{man(7), groff(7), man-pages(7)} \\
										8 & System administration commands (usually only for \texttt{root}) \\
										9 & Kernel routines [Non standard]
									\end{tabular}
								\end{table}
					\end{itemize}
				\item \mintinline{bash}{pwd} \output{output/pwd}
					\begin{itemize}
						\item Print the full filename of the current working directory
						\item It avoids all the symlinks and displays the directory if the \option{-P}
					\end{itemize}
				\item \mintinline{bash}{od /dev/null} \output{output/od}
					\begin{itemize}
						\item dump files in octal and other formats
						\item Write  an  unambiguous representation, octal bytes by default, of FILE to standard
							output
						\item With more than one FILE argument, concatenate them in the listed order to form the
							input
					\end{itemize} \newpage
				\item \mintinline{c}{cal 1 1970} \output{output/cal}
					\begin{itemize}
						\item The \option{cal} utility displays a simple calendar in traditional format, more
							options and the date of Easter.
						\item The new format is a little cramped but it makes a year fit on a $25\times80$
							terminal.
						\item If arguments are not specified, the current month is displayed.
					\end{itemize}
				\item \mintinline{c}{date --date="1/1/70" +"%tToday is %A, %d %B %Y"} \output{output/date}
				\begin{itemize}
					\item Display  date  and  time  in the given FORMAT.
					\item With \option{-s}, or with \mintinline{bash}{[MMDDhhmm[[CC]YY][.ss]]}, set the date \&
						time.
					\item Mandatory arguments to long options are mandatory for short options too.
				\end{itemize}
			\end{enumerate} \newpage
		\question{1b}{%
			With a neat diagram, explain the kernel and shell relationship in UNIX operating System
		}{~}
		\answer \def \tn#1#2{node[draw=white,minimum width=#1,minimum height=#2]}
\def \trn#1#2#3{node[#1,draw=white,minimum width=#2,minimum height=#3]}

\begin{figure}[H]
	\centering
	\begin{tikzpicture}[node distance=5em]
		\draw
			\tn{5em}{2em} (u) {User}
			\trn{below of=u}{5em}{2em} (sh) {Shell}
			\trn{below of=sh,xshift=-10em}{5em}{2em} (ut) {Utilities}
			\trn{below of=sh,xshift=10em}{5em}{2em} (ap) {
				\begin{tabular}{c}Application \\ Program\end{tabular}
			}
			\trn{below of=sh,yshift=-5em}{5em}{2em} (k) {Kernel}
			\trn{below of=k}{5em}{2em} (h) {Hardware}

		;
		\draw[-Straight Barb] (u) -- (sh);
		\draw[-Straight Barb](sh) -| (ut);
		\draw[-Straight Barb](sh) -| (ap);
		\draw[-Straight Barb](sh) -- (k);
		\draw[-Straight Barb](k) -- (h);
	\end{tikzpicture}
\end{figure}

			\begin{itemize}
				\item Kernel
					\begin{itemize}
						\item It is the heart of UNIX system.
						\item It contains two basic parts of the OS, process control and resource management.
						\item All other components of the system call on the kernel to perform these services
							for them.
					\end{itemize}
				\item Shell
					\begin{itemize}
						\item Interface between Kernel and User.
						\item It functions as command interpreter i,e it receives and interprets the command
							from user and interacts with the hardware.
						\item There is only one kernel running on the system, there could be several shells in
							action one for each user who is logged in.
						\item Shell has two major parts.
							\begin{enumerate}[label=\alph*)]
								\item Interpreter \\
									It reads your commands and works with the kernel to execute them.
								\item Shell Programming
									\begin{itemize}
										\item It is a programming capability that allows you to write a shell
											scripts.
										\item It is also known as shell program.
										\item A shell script is a file that contains the shell commands that
											perform a useful function
										\item There are three standard shells used in UNIX today.
											\begin{enumerate}
												\item Bourne shell \\
													Developed by steve bourne at AT\&T labs, is the oldest.
												\item Bash (Bourne Again shell) \\
													An enhanced version of the Bash shell.
												\item C shell \\
													Developed in the Berkeley by Bill joy, Its commands look
													like C statements.
												\item Tcsh \\
													A compatible version of C shell.
												\item Korn shell \\
													Developed by David Korn, also of AT\&T Labs is the newest
													and powerful.
											\end{enumerate}
									\end{itemize}
							\end{enumerate}
					\end{itemize}
			\item Utilities
				\begin{itemize}
					\item A utility is a standard Unix program that provides a support for users.
					\item Three common utilities are text editors, search programs and sort programs.
				\end{itemize}
			\item Applications
				\begin{itemize}
					\item They are programs that are not a standard part of UNIX.
					\item Written by system administrator's professional programmers or users they provide an
						extended capability to the system.
				\end{itemize}
			\end{itemize}
		\OR
		\question{2a}{Explain the salient features of Unix Operating System}{~}
		\answer Several features of UNIX have made it popular. Some of them are :
			\begin{itemize}
				\item Portable
					\begin{itemize}
						\item UNIX can be installed on many hardware platforms.
						\item Its widespread use can be traced to the decision to develop it using the C
							language.
						\item Because C programs are easily moved from one hardware environment to another, it
							is relatively simple to port it to different environments.
					\end{itemize}
				\item Multiuser
					\begin{itemize}
						\item The UNIX design allows multiple users to concurrently share hardware and software
					\end{itemize}
				\item Multitasking
					\begin{itemize}
						\item UNIX allows a user to run more than one program at a time.
						\item In fact more than one program can be running in the background while a user is
							working foreground.
					\end{itemize}
				\item Networking
					\begin{itemize}
						\item While UNIX was developed to be an interactive, multiuser, multitasking system,
							networking is also incorporated into the heart of the operating system.
						\item Access to another system uses a standard communications protocol known as
							Transmission Control Protocol/Internet Protocol (TCP/IP).
					\end{itemize}
				\item Organized File System
					\begin{itemize}
						\item UNIX has a very organized file and directory system that allows users to organize
							and maintain files.
					\end{itemize}
				\item Device Independence
					\begin{itemize}
						\item UNIX treats input/output devices like ordinary files.
						\item Input or output to a program can be from any device or file.
						\item The source or destination for file input and output is easily controlled through a
							UNIX design feature called redirection.
					\end{itemize}
				\item Utilities
					\begin{itemize}
						\item UNIX provides a rich library of utilities that can be use to increase user
							productivity.
					\end{itemize}
				\item Services
					\begin{itemize}
						\item UNIX also includes the support utilities for system administration and control.
					\end{itemize}
			\end{itemize}
		\question{2b}{%
			Differentiate between Internal and External commands in UNIX operating system with example
		}{~}
		\answer UNIX commands are classified into two types
			\begin{itemize}
				\item Internal Commands
					\begin{itemize}
						\item Internal commands are something which is built into the shell.
						\item For the shell built in commands, the execution speed is really high.
						\item It is because no process needs to be spawned for executing it.
						\item Example : \mintinline{bash}{echo}, \mintinline{bash}{cd}
						\item[] \texttt{\$} \mintinline{bash}{type echo cd} \\
								\texttt{echo is a shell builtin \\ cd is a shell builtin}
					\end{itemize}
				\item External Commands
					\begin{itemize}
						\item External commands are not built into the shell.
						\item These are executable present in a separate file.
						\item When an external command has to be executed, a new process has to be spawned and
							the command gets executed.
						\item Example : \mintinline{bash}{ls}, \mintinline{bash}{cat}
						\item[] \texttt{\$} \mintinline{bash}{type ls cat} \\
								\texttt{ls is /usr/bin/ls \\ cat is /usr/bin/cat}
				\def \excmd{
					\mintinline{bash}{clear_history is aliased to `exec rm -rf .*_history .lesshst .wget-hsts'}
				}
						\item[] \texttt{\$} \mintinline{bash}{clear_history} \\ \excmd
					\end{itemize}
			\end{itemize}
		\OR
		\question{3a}{Explain ~\mintinline{bash}{ls}~ command with all the options}{~}
		\answer The \verb|ls| command lists all files in the directory that match the name. \\
			If name is left blank, it will list all of the files in the directory. \\ ~ \\
			\textbf{Syntax} \\ \verb|ls [options] [names]|
			\begin{table}[H]
				\centering
				\begin{tabular}{c|l}
					Option & \multicolumn{1}{|c}{Description} \\ \hline
					\option{-a} & Displays all files \\
					\option{-b} & Displays non-printing characters in octal \\
					\option{-c} & Displays files by file time-stamp \\
					\option{-C} & Displays files in a columnar format (default) \\
					\option{-d} & Displays only directories \\
					\option{-f} & Interprets each name as a directory, not a file \\
					\option{-F} & Flags filenames \\
					\option{-g} & Displays the long format listing, but exclude the owner name \\
					\option{-i} & Displays the inode for each file \\
					\option{-1} & Displays the long format listing \\
					\option{-L} & Displays the file or directory referenced by a symbolic link \\
					\option{-m} & Displays the names as a comma-separated list \\
					\option{-n} & Displays the long format listing, with GID and UID numbers \\
					\option{-o} & Displays the long format listing, but excludes group name \\
					\option{-p} & Displays directories with / \\
					\option{-q} & Displays all non-printing characters as ? \\
					\option{-r} & Displays files in reverse order \\
					\option{-R} & Displays subdirectories as well \\
					\option{-t} & Displays newest files first. (based on time-stamp) \\
					\option{-u} & Displays files by the file access time \\
					\option{-x} & Displays files as rows across the screen \\
					\option{-1} & Displays each entry on a line
				\end{tabular}
			\end{table}
		\question{3b}{Define wild cards ? Explain various shell wild cards with suitable example}{~}
		\answer \textbf{Wild Cards} \\
			The meta-characters that are used to construct the generalized pattern for matching filenames
			belong to the category called wild cards.
			\begin{itemize}
				\item Asterisks \verb|[*]|
					\begin{itemize}
						\item The meta-character \verb|*| is one of the characters of the shell wild card set
						\item It matches any number of characters including none
						\item Example :
							\begin{itemize}
								\item \mintinline{bash}{ls *.pdf *.csv         # list pdf and csv files}
								\item \mintinline{bash}{convert *.png anim.gif # create a gif from images}
							\end{itemize}
					\end{itemize}
				\item Question mark \verb|[?]|
					\begin{itemize}
						\item It matches a single character like alphanumeric and special characters in the
							ascii table
						\item Example :
							\begin{itemize}
								\item \mintinline{bash}{rename 's/^draft//' draft?.tex   # remove prefix draft}
								%\item \mintinline{bash}{rename 's/.docx$/.pdf/' hi?.docx # change extension}
							\end{itemize}
					\end{itemize}
				\item Dot [$\cdot$]
					\begin{itemize}
						\item The dot metachacter can be used to match all the hidden files in your directory
						\item Example
							\begin{itemize}
								\item \verb|$| \mintinline{bash}{ls ?.*}
								\item \verb|$| \mintinline{bash}{stat ??.???}
							\end{itemize}
					\end{itemize} \newpage
				\item The character class \verb|([ ])|
					\begin{itemize}
						\item This class comprises a set of charcters enclosed by the rectangular brackets
							\verb|[and]|, but it matches only a single character in the class.
						\item The pattern \verb|[abed]| is a character class and it matches a single character
							\verb|a, b, c| or \verb|d|
						\item Negating character class is done by using \verb|[!]|
						\item Example
							\begin{itemize}
								\item \verb|$| \mintinline{bash}{ls [0-9][a-z]??.txt}
								\item \verb|$| \mintinline{bash}{cat *.[!oc]}
							\end{itemize}
					\end{itemize}
			\end{itemize}
		\OR
		\question{4a}{Explain changing file permissions in relative and absolute manner}{~}
		\answer A file or a directory is created with a default s t of permissions, which can be detennin d by
			umask 
			\begin{enumerate}[label=\roman*)]
				\item Relative Permission
					\begin{itemize}
						\item \verb|chmod| only changes the permissions specified in the command line and leaves
							the other permissions unchanged.
						\item \textbf{syntax} \\
							\verb|$ chmod category operation permission file(s)|
						\item The below shows the abbreviations used by \verb|chmod| command
							\begin{table}[H]
								\centering
								\begin{tabular}{l|l|l}
									Category & Operation & Permission \\ \hline
									u - user & + assign & r - read \\
									g - group & - remove & w - write \\
									o - other & = absolute & x - execute \\
									a - (ugo) & &
								\end{tabular}
							\end{table}
					\end{itemize}
				\item Absolute Permissions
					\begin{itemize}
						\item A string of three octal digits is used as an expression.
						\item The permission can be represented by one octal digit for each category.
						\item For ach cat gory, we add octal digits.
						\item If we represent the permission of each category by one octal digit, this is how
							the permission can be represented :
							\begin{table}[H]
								\centering
								\begin{tabular}{c|c}
									Permission & Value \\ \hline
									Read & 4 (octal 100) \\
									Write & 2 (octal 010) \\
									Execute & 1 (octal 001)
								\end{tabular} \\ ~\vspace{1em} \\
								\begin{tabular}{c|c|l}
									Octal & Permissions & Significance \\ \hline
									0 & \verb|---| & no permissions \\
									1 & \verb|--x| & execute only \\
									2 & \verb|-w-| & write only \\
									3 & \verb|-wx| & write and execute \\
									4 & \verb|r--| & read only \\
									5 & \verb|r-x| & read and execute \\
									6 & \verb|rw-| & read and write \\
									7 & \verb|rwx| & read, write and execute
								\end{tabular}
							\end{table}
					\end{itemize}
			\end{enumerate}
		\question{4b}{%
			Explain ~\mintinline{bash}{if}~ and ~\mintinline{bash}{while}~ control statements in shell scripts
			with suitable program
		}{~}
		\OR
		\question{5a}{ Explain the following functions with their prototypes and examples}{~}\!:
			\begin{enumerate}[label=\roman*)]
				\item \mintinline{c}{mkdir ()}
				\item \mintinline{c}{rmdir ()}
				\item \mintinline{c}{fcntl ()}
			\end{enumerate}
		\question{5b}{Discuss the various UNIX system implementation}{~}
		\OR
		\question{6a}{With a neat diagram explain how a C program is started and terminated}{~}
		\question{6b}{Explain ~\mintinline{c}{getrlimit ()}~ and ~\mintinline{c}{setrlimit ()}~ APIs}{~}
		\OR
		\question{7a}{
			Explain the ~\mintinline{c}{fork ()}~ and ~\mintinline{c}{vfork ()}~ system calls \\
			How does ~\mintinline{c}{fork ()}~ differ from ~\mintinline{c}{vfork ()}~ ?
		}{%
			Explain the ~\mintinline{c}{fork ()}~ and ~\mintinline{c}{vfork ()}~ system calls.
			How does ~\mintinline{c}{fork ()}~ differ from ~\mintinline{c}{vfork ()}~ ?
		}
		\question{7b}{Write short notes on the following with examples}{~}\!:
			\begin{enumerate}[label=\roman*)]
				\item \mintinline{c}{exec ()}
				\item Race conditions
			\end{enumerate}
		\OR
		\question{8a}{
			What are pipes ? What are its limitations ? \\
			Write a program to send data from parent to child over a pipe.
		}{%
			What are pipes ? What are its limitations ?
			Write a program to send data from parent to child over a pipe.
		}
		\question{8b}{What is a FIFO ? With a neat diagram explain client server communication using FIFO.}{~}
		\OR
		\question{9a}{Define Signal ? Explain ~\mintinline{c}{sigprocmask ()}~ function with a program}{~}
		\question{9b}{Explain Daemon characteristics and coding rules}{~}
		\OR
		\question{\!\!10a}{What is a daemon process ? Explain coding rules and error logging}{~}
		\question{\!\!10b}{Explain the following functions}{~}\!:
			\begin{enumerate}[label=\roman*)]
				\item \mintinline{c}{system ()}
				\item \mintinline{c}{sleep ()}
			\end{enumerate}
		\answer
			\begin{enumerate}[label=\roman*)]
				\item \mintinline{c}{system ()}
					\begin{itemize}
						\item ISO C defines the system function, but its operation is strongly system dependent
						\item POSIX.1 includes the system interface, expanding on the ISO C definition to
							describe its behavior in a POSIX environment.
						\item \textbf{syntax} \inputminted{c}{functions/system.c}
						\item If \verb|cmdstring| is a null pointer, system returns nonzero only if a command
							processor is available.
						\item This feature determines whether the system function is supported on a given
							operating system.
						\item The \mintinline{c}{system ()} is implemented by calling
							\mintinline{c}{fork ()}, \mintinline{c}{exec ()}, and \mintinline{c}{waitpid ()}
							there are three types of return values.
							\begin{itemize}
								\item If either the \mintinline{c}{fork ()} fails or \mintinline{c}{waitpid ()}
									returns an error other than \verb|EINTR|, \mintinline{c}{system ()} returns
									-1 with \verb|errno| set to indicate the error.
								\item If the \mintinline{c}{exec ()} fails, implying that the shell can't be
									executed, the return value is as if the shell had executed
									\mintinline{c}{exit (127)}.
								\item Otherwise, all three functions \mintinline{c}{fork ()},
									\mintinline{c}{exec ()}, and \mintinline{c}{waitpid ()} succeed, and the
									return value from \mintinline{c}{system ()} is the termination status of the
									shell, in the format specified for \mintinline{c}{waitpid ()}.
						\end{itemize}
					\end{itemize}
				\item \mintinline{c}{sleep ()}
					\begin{itemize}
						\item This function causes the calling process to be suspended until either
							\begin{itemize}
								\item The amount of wall clock time specified by seconds has elapsed
								\item A signal is caught by the process and the signal handler returns
							\end{itemize}
						\item \textbf{syntax} \inputminted{c}{functions/sleep.c}
						\item As with an alarm signal, the actual return may occur at a time later than
							requested because of other system activity
						\item The \mintinline{c}{nanosleep ()} function is similar to the sleep function, but
							provides nanosecond-level granularity.
						\item \textbf{syntax} \inputminted{c}{functions/nanosleep.c}
						\item This function suspends the calling process until either the requested time has
							elapsed or the function is interrupted by a signal
						\item The reqtp parameter specifies the amount of time to sleep in seconds and
							nanoseconds
						\item If the sleep interval is interrupted by a signal and the process doesn't
							terminate, the \verb|timespec| structure pointed to by the \verb|remtp| parameter
							will be set to the amount of time left in the sleep interval
						\item \textbf{syntax} \inputminted{c}{functions/clock_nanosleep.c}
						\item A way to suspend the calling thread using a delay time relative to a particular
							clock
						\item The \verb|clock_id| argument specifies the clock against which the time delay is
							evaluated.
						\item The \verb|flags| argument is used to control whether the delay is absolute or
							relative.
						\item When flags is set to 0, the sleep time is relative
						\item When it is set to \verb|TIMER_ABSTIME|, the sleep time is absolute
					\end{itemize}
			\end{enumerate}
	\end{enumerate} \newpage
	\vspace*{\fill} \thispagestyle{empty}
			\centering \emph{This page is left blank intentionally}
	\vfill
\end{document}
