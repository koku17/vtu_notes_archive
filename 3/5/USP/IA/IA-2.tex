\documentclass{article}

%! TEX program=xelatex

% packages
\usepackage[margin=.5in]{geometry}
\usepackage{xcolor}
\usepackage{enumitem}
\usepackage[outputdir=tmp,cache=false]{minted}
\usepackage{hyperref}
\usepackage{pgffor}

% detailes
\author{koku17}
\title{IA-2 Important QA}

% presets
\hypersetup{
	hidelinks
}
\setminted{	
	obeytabs=true,
	tabsize=4
}

% ricing
\def \darkmode{1}

\if\darkmode1
	\pagecolor{black}
	\color{white}
	\usemintedstyle{one-dark}
\else
	\usemintedstyle{perldoc}
\fi

% macro
\newcommand{\module}[1]{
	\begin{center}
		\phantomsection \Large Module #1
	\end{center}
	\addcontentsline{toc}{section}{Module #1}
}

\newcommand{\question}[2]{
	{\phantomsection \item #1}
	\ifx #2
		\addcontentsline{toc}{subsection}{~#1}
	\else
		\addcontentsline{toc}{subsection}{#2}
	\fi
}

\def \answer {\item[$\rightarrow$]}

\def \output {
	\begin{center}
		\phantomsection \textbf{Output}
		\addcontentsline{toc}{subsubsection}{Output}
	\end{center}
}

\begin{document}
	\pagenumbering{gobble} \maketitle \newpage
	\pagenumbering{roman} \pdfbookmark[1]{Contents}{} \tableofcontents \newpage
	\pagenumbering{arabic}

	\module{2}
	\begin{enumerate}[label=\arabic*)]
		\question{
			Write a program to read pattern \& filename from the user and search the pattern in the given file
		}{~Search the pattern in the given file by user}
		\answer \inputminted{sh}{matchpat.sh}
		\output \verb|$ sh matchpat.sh| \inputminted{text}{1.out} \newpage
		
		\question{Explain all the looping statements with syntax}{ }
		\answer Looping statements are :
			\begin{enumerate}[label=\roman*)]
				\item \mint{bash}{for}
					\begin{itemize}
						\item The \mintinline{bash}{for} loop operates on lists of items.
						\item It repeats a set of commands for every item in a list.
						\item \textbf{Syntax} \inputminted{bash}{for_syntax.sh}
						\item \textbf{Example} \inputminted{bash}{for_example.sh}
						\item \textbf{Output} \\ \texttt{\foreach \i in {1,...,5} {\i~}}
					\end{itemize}
				\item \mint{bash}{while}
					\begin{itemize}
						\item A \mintinline{bash}{while} loop uses a test condition to determine whether or not
							to execute it's command block.
						\item If condition returns true when tested, the loops executes commands or if it
							returns false, the script skips the loop and proceeds beyond its terminate point
						\item Because the loop starts with the test, if the condition fails, then the program
							continues at whatever action immediately follows the loops done statement
						\item If the condition passes then the script executes the enclosed command block
						\item After performing the last command in the block, the loop starts tests the
							condition again \& determine whether to proceed beyond the loop or fail through it
							once more
						\item \textbf{Syntax} \inputminted{bash}{while_syntax.sh}
						\item \textbf{Example} \inputminted{bash}{while_example.sh}
						\item \textbf{Output} \\ \texttt{\foreach \i in {1,...,5} {\i~}}
					\end{itemize} \newpage
				\item \mint{bash}{until}
					\begin{itemize}
						\item The condition is evaluated at the start of each iteration.
						\item If the condition is false, the block of statements inside the loop is executed.
						\item After executing the block, the loop returns to check the condition again.
						\item The loop continues until the condition becomes true.
						\item Once the condition is true, the control moves to the next command after the loop.
						\item \textbf{Syntax} \inputminted{bash}{until_syntax.sh}
						\item \textbf{Example} \inputminted{bash}{until_example.sh}
						\item \textbf{Output} \\ \texttt{\foreach \i in {5,...,1} {\i~}}
					\end{itemize}
			\end{enumerate} \newpage

		\question{
			Write a shell script to read multi-file patterns from command line and search these patterns in a
			given file which is also read from command line
		}{~Write a shell script to read multiple patterns from a given file by command line arguments}
		\answer
			\inputminted{bash}{multi-pattern.bash}
		\output \verb|$ bash multi-pattern.bash author doc IA-2.tex| \inputminted{text}{3.out} \newpage

		\question{
			Define Shell Script, write a menu-driven shell script which display the following
		}{~Define Shell Script, write a menu-driven shell script which displays the information}
			\begin{enumerate}[label=\roman*)]
				\item Current User
				\item List of files
				\item Today's date
				\item Process status
				\item Contents of file
			\end{enumerate}
		\answer \textbf{Shell Script} \\
			When a shell is executing non-interactively, the commands executed are read from file. This method
			of using files for execution instead of manual typing by keyboard interactively in a shell is known
			as shell scripting and the file is known as shell script
		\inputminted{bash}{menu.sh} \newpage
		\output \verb|$ sh menu.sh| \inputminted{text}{4.out} \newpage

		\question{Define a Shell. Briefly explain the shell interpretation cycle}{ }
		\answer \textbf{Shell}
			\begin{itemize}
				\item A shell is a program that acts as the interface between user \& the system, allowing user
					to enter commands for the operating system to execute.
				\item Shell is both a command interpreter and a programming.
				\item As a command interpreter, the shell provides the user interface to the set of utilities.
				\item The programming language features allow these utilities to be combined.
				\item Files containing commands can be created and become commands themselves.
				\item These new commands have the same status as system commands in directories such as /bin,
					allowing users or groups to establish custom environments to automate their commands tasks.
			\end{itemize}

			\textbf{Shell interpretation cycle}
			\begin{itemize}
				\item Shells may be used interactively or non-interactively, in interactive mode, they accept
					input typed from the keyboard.
				\item When executing non-interactively, shells execute commands read from a file.
				\item A shell allows execution of commands both synchronously \& asynchronously.
				\item The shell waits for synchronous commands to complete before accepting more input,
					asynchronously commands continue to execute in parallel with the shell while it reads \&
					executes additional commands.
			\end{itemize} \newpage

		\question{Explain logical operators and conditional operators in Shell script}{ }
		\answer \textbf{The Logical Operators}
			\begin{itemize}
				\item When you evaluate expressions we need a \mintinline{bash}{test} statement because the true
					or false values returned by expressions can't be directly handled by \mintinline{bash}{if}
				\item Shell provides two operators that allow conditional execution -- \mintinline{bash}{&&} and
					\mintinline{bash}{||}
				\item \textbf{Example} \inputminted{bash}{logical_example}
			\end{itemize}

			\textbf{Conditional operators}
			\begin{itemize}[label=]
				\item \textbf{\mintinline{bash}{if}-conditional}
					\begin{itemize}
						\item The if statement makes two-way decisions depending on the fulfillment of a certain
							condition shown as below \inputminted{bash}{conditional_if}
						\item \textbf{Example} \inputminted{bash}{if_examples}
					\end{itemize} \newpage
				\item \textbf{\mintinline{bash}{case}-conditional}
					\begin{itemize}
						\item Case conditional is a special case of \mintinline{bash}{if}\verb|-else| ladder
						\item It accepts multiple patterns such as regular expressions
						\item If the variable value matches then it executes the command block
						\item Each command block ends with \mintinline{bash}{;;}, after reaching command block
							shell terminates in the end denoted by \mintinline{bash}{esac}
						\item \textbf{Syntax} \inputminted{bash}{conditional_case}
						\item \textbf{Example} \inputminted{bash}{conditional_examples}
					\end{itemize}
			\end{itemize} \newpage

		\question{Write a shell script to compare two numbers and print the result}{ }
		\answer \inputminted{bash}{cmp.sh}
		\output
			\begin{enumerate}[label=\arabic*)]
				\item \verb|$ sh cmp.sh 1 2| \inputminted{text}{8.1.out}
				\item \verb|$ sh cmp.sh 2 2| \inputminted{text}{8.2.out}
			\end{enumerate} \newpage

		\question{Explain the \mintinline{bash}{grep} command with all it's options}{ }
		\answer \textbf{Options} \inputminted{text}{grep-options}
			\textbf{Examples} \inputminted{bash}{grep-examples} \newpage

		\question{Explain 3 Standard files with respect to Unix OS}{ }
		\answer
			\begin{itemize}
				\item \textbf{Standard input} \\
					This file stream representing input, connected to the keyboard. \\
					The standard input can represent three input sources :
					\begin{itemize}
						\item The keyboard, the default source.
						\item A file using redirection with the \mintinline{bash}{<} symbol.
						\item Another program using a pipeline \mintinline{bash}{|} symbol.
					\end{itemize}
				\item \textbf{Standard output} \\
					This file stream representing output, connected to the display. \\
					The standard output can represent three possible destinations : 
					\begin{itemize}
						\item The terminal, the default destination.
						\item A file using the redirection symbols \mintinline{bash}{>} and
							\mintinline{bash}{>>}.
						\item As input to another program using a pipeline.
					\end{itemize}
				\item \textbf{Standard error} \\
					This file stream representing error messages that emanate from the command or shell,
					connected to the display.
				\item A file is opened by referring to its pathname, but subsequent read and write operations
						identify the file by a unique number called a file descriptor
				\item The kernel maintains a table of file descriptors for every process running in the system
				\item The first three slots are generally allocated to the three standard streams as,
					\begin{enumerate}[label=\arabic* -,start=0]
						\item Standard input
						\item Standard output
						\item Standard error
					\end{enumerate}
			\end{itemize}
	\end{enumerate} \newpage

	\module{3}
	\begin{enumerate}[label=\arabic*)]
		\question{Write C-POSIX complaint program to check the following values}{ }
			\begin{enumerate}[label=\roman*)]
				\item Number of clock ticks
				\item Maximum path length
				\item Maximum number of child processes
				\item Number of open files or process
			\end{enumerate}
		\answer \inputminted{c}{posix_max.c}
		\output \inputminted{text}{max.out}
	\end{enumerate}
\end{document}
